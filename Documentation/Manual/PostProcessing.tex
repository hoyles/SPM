\section{Post processing output using \R \label{sec:post-processing}}

The \R\ package \texttt{spm} contains a set of \R\ functions for reading \SPM\ output, and is available as a precompiled binary for Microsoft Windows (.zip file) or as a source package (.gz file) for Linux. To check the version number and date of the \texttt{spm} \R\ package (useful for checking that you have the most recent version), use the function \texttt{spm.version()}.

The \texttt{spm} \R\ package includes a range of extract and write functions to aid post-processing of \SPM\ \config s and output. The main extract functions are briefly described below. In addition, the package also some undocumented helper functions, that could be useful for writing you own analysis functions. See the \R\ help for more detail e.g., \texttt{help(spm)}

\subsection{Read and extract reports from a SPM output file.}

Command: \texttt{extract()} 

Usage: \texttt{extract(file, path = "", ignore.unknown=FALSE)}

Arguments:
\begin{description}
\item[\texttt{file}] the name of the SPM output file to read
\item[\texttt{path}] Optionally, the operating system path to the directory of the output file.
\item[\texttt{ignore.unknown}] Ignore unknown reports when reading. (This can be useful to read files that contain undocumented reports or other output)
\end{description}

Output: A list object with elements for each report type.

\subsection{Read and write \SPM\  config files.}

Command: \texttt{spm.read.config()} 

Usage: \texttt{extract(file, path = "")}

Arguments:
\begin{description}
\item[\texttt{file}] the name of the SPM config file to read
\item[\texttt{path}] Optionally, the operating system path to the directory of the output file.
\end{description}

Output: A list object with elements for each command block in the config file.

Command: \texttt{spm.write.config()} 

Usage: \texttt{write.spm.config(object, file, path = "", header = "SPM config file", date = FALSE)}

Arguments:
\begin{description}
\item[\texttt{object}] the name of the list object to write as an SPM config file
\item[\texttt{file}] the name of the SPM config file to write to
\item[\texttt{path}] Optionally, the operating system path to the directory of the output file.
\item[\texttt{header}] Comment at the head of the output file with a user defined description.
\item[\texttt{date}] Optionally, write a comment giving the system date time that the file was created.

\end{description}

Output: A SPM config file.
